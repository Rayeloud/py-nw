% In the context of this thesis, a numerical model has been developed to first assess the relevance of the phase-field modeling theory in the study of the morphological instability of metallic nanowires, and then to study the morphological instability itself in a non-dimensionalised manner. The thesis started by introducing the context and key theoretical concepts in \autoref{chap:1-sota}, and the the methodology employed to achieve the aim was detailed in \autoref{chap:2-methodology}. Then the 
% \section*{Perspectives}
% %\lipsum[1]
The increasing demand for transparent electronics has stimulated research into alternatives to traditional transparent conducting materials (TCMs). Silver nanowire (AgNW) networks have emerged as promising candidates thanks to their excellent electrical and optical properties. However, their limited thermal stability, manifested as morphological instability and breakup of nanowires at elevated temperatures, remains a major obstacle to their practical application.

The core objective of this thesis was to, first, further assess the relevance of phase-field modeling as a tool to study the morphological instability of metallic nanowires, and to develop a predictive, numerical framework to understand the breakup mechanism of metallic nanowires and propose potential optimization strategies for stabilizing nanowire networks. Experimental studies have offered qualitative insights, but their limitations in capturing and generalizing the underlying mechanisms necessitate the use of robust numerical models. In this context, the phase-field method has gained traction as a flexible and thermodynamically consistent tool for modeling microstructural physics, and especially interface-driven morphological evolutions.

To tackle this problem, a phase-field framework based on the Cahn-Hilliard equation was adopted. The numerical implementation, introduced in \autoref{chap:2-methodology}, involved a Fourier spectral semi-implicit scheme. The solver was validated through numerical experiments to ensure consistency and convergence. A key extension in \autoref{chap:4-substrate} involved the consideration of the Smoothed Boundary Method (SBM), allowing a diffuse representation of complex geometries and enforcement of boundary conditions.

In \autoref{chap:3-results}, the framework was used to revisit and extend the earlier work of Roy et al. on nanowire breakup. The study focused on two geometrical configurations, a single free-standing nanowire and a junction of two free-standing nanowires. The results lead to the identification of power-law scaling relationships for the breakup time and the instability wavelength, consistent with theoretical predictions. In addition, a disentanglement of the two dominant breakup mechanisms was achieved: perturbation-driven, governed by the initial perturbations, and free-end-driven, initiated from the free ends of the nanowire. The results found the wavelength of the instability in finite nanowires to be smaller than the maximally growing wavelength as predicted by classical theory. An extension to more realistic geometries was also performed, revealing accelerated breakup due to the initial faceting of the nanowire. 

In \autoref{chap:4-substrate}, the model was generalized to study substrate-supported nanowires using the Smoothed Boundary Method. The presence of the substate was shown to significantly alter the dynamics of the morphological instability, leading to slower dynamics and modified wavelength of instability. The observed effects were in agreement with the theoretical predictions by McCallum et al.~. A comparative analysis between different prescribed contact angles revealed that the presence of the substrate does not alter the scaling laws derived for the free-standing case but instead modifies the instability landscape. 

The findings of this thesis establish the phase-field method as a robust and flexible tool for studying surface-driven morphological instabilities in metallic nanowires. The consistent reproduction of known scaling laws across various configurations validates its application to nanoscale systems and bridges a critical gap between theory, simulation, and experiment.

While the presented model captures the essential physics, the study also highlights limitations and room for improvement. Notably, the simplified isotropic surface energy approximation neglects crystallographic anisotropy, which is expected to play a substantial role in real systems. Furthermore, the absence of elastic effects, temperature-dependent diffusion parameters, and grain-boundary considerations limits the model's applicability to complex annealing scenarios.

\subsection*{Perspectives for Future Work}
Beyond the current scope, several promising directions could be explored to further enhance the physical accuracy and applicability of the phase-field model. Coupling the Cahn-Hilliard equation with the Allen-Cahn equation under the Kim-Kim-Suzuki (KKS) formalism would allow tracking of the individual grains composing the nanowire, thereby enabling grain-boundary-driven phenomena to be captured explicitly. Such an approach also permits anisotropic behavior to be modeled via the non-conserved phase field specific mobility $L$, enabling crystallographic orientation-dependent kinetics.

An alternative strategy for incorporating anisotropy is through the explicit definition of a surface energy density that depends on the interface normal, directly within the Cahn-Hilliard formulation. This would allow directional surface energy minimization effects to be accounted for without introducing an additional field variable.

Moreover, the phase-field formalism can be extended to multiphysics contexts by incorporating the correct energetic contributions of various physical fields like elastic strain, temperature, electric fields, and even magnetic interactions. Recent developments have introduced phase-field frameworks that couple electro-thermo-mechanical fields for modeling metallic interconnects. This approach could be adapted to model the interplay between thermal, mechanical and morphological effects in nanowire networks.

Nevertheless, the phase-field method remains computationally intensive, especially when targeting realistic 3D morphologies over long physical timescales. Future work may also investigate optimization strategies aimed at reducing computational overhead.
