\section{Phase field modelling of a binary mixture}
    The evolution of the composition of a binary, ternary, n-ary mixture can be studied thanks to the Cahn-Hilliard equation which describes the evolution of a particle in the system as such
    \begin{equation}
        \frac{\partial c}{\partial t} = M \nabla^2 \mu 
    \end{equation}
    where $c$ is a non-dimensional description of the composition of a particle bounded between 0 and 1, $\mathbf{M} = M \delta_{ij}$ is the mobility tensor (isotropic in this case) and $\mu$ is the chemical potential of the particle.

    The chemical potential is defined as the variational derivative of the Gibbs free energy:
    \begin{equation}
        \mu = \frac{1}{N_v} \frac{\delta F}{\delta c}
    \end{equation}
    where the Gibbs free energy function of an \textbf{isotropic} binary mixture is described as follows: % cite properly cahn and hilliard paper
    \begin{equation}
        F(c) = N_v\int_\Omega f_0 + \frac{\kappa}{2} |\nabla c|^2 + \dots\,d\Omega
    \end{equation}
    Performing the variational derivative yields the following expression of the chemical potential:
    \begin{equation}
        \mu = \frac{\partial f_0}{\partial c} - \kappa \Delta c + \dots
    \end{equation}
    The evolution of the conserved variable $c$ writes:
    \begin{equation}
        \frac{\partial c}{\partial t} = \Vec{\nabla}\left[M \Vec{\nabla}\left(\frac{\partial f_0}{\partial c} - \kappa\Delta c \right) \right]
    \end{equation}
    To capture the interface-driven diffusion that takes place in the case of nanowire breakup dynamics, the mobility is defined as a function of the composition field:
    \begin{equation}
        M=(1-ac^2)
    \end{equation}
    where $a=0$ represents bulk-diffusion controlled dynamics and $a=1$ represents interface-diffusion controlled dynamics (where $c=\pm1$ are the pure composition.).\\
    The bulk free energy $f_0$ of the binary mixture is described as a double-well potential:
    \begin{equation}
        f_0 = Ac^2(1-c^2)
    \end{equation}
    with $A$ an additional parameter which represents the height of the barrier.\\
    Thus, the evolution of the composition field writes:
        \begin{equation}
        \frac{\partial c}{\partial t} = \Vec{\nabla}\left[(1-ac^2) \Vec{\nabla}\left(g(c) - \kappa\Delta c \right) \right]
    \end{equation}
    where $g(c)=\frac{\partial f_0}{\partial c}$.
    \section{Numerical method}
    In order to solve the equation numerically, the Spectral Fourier method scheme is used instead of finite-difference methods. The method transforms the partial differential equation into a sequence of ordinary differential equations in the Fourier space:
    \begin{equation}
        \frac{\partial \hat{c}}{\partial t} = j\mathbf{k} \left\{ \phi (1-ac^2) \left[ j\mathbf{k}' (\hat{g} + \kappa {k'}^2 \hat{c}) \right]_r \right\}_k
    \end{equation}
    where $\mathbf{k}$ is a vector in the Fourier space and $k$ is its magnitude, $\hat{(.)}$ represents a quantity in the Fourier space. $\{.\}_k$ and $[.]_r$  represents respectively the forward and inverse Fourier transform of the quantity in, the square brackets and the curly brackets. 
    The first order differential equation can be solved using the forward Euler explicit method:
    \begin{equation}
        \hat{c}^{t+\Delta t}-\hat{c}^{t} = j\Delta t\mathbf{k} \left\{ (1-ac^2) \left[ j\mathbf{k}' (\hat{g} + \kappa {k'}^2 \hat{c}) \right]_r^t \right\}_k
    \end{equation}
    However, the considered scheme has a very restricting constraint on the time discretization. To alleviate the constraint on it, a semi-implicit treatment is performed on the explicit scheme. To do so, the mobility term is split into two parts: $\xi$ and $(1-ac^2) - \xi$. This treatment yields the following:
    \begin{equation}
        \hat{c}^{t+\Delta t}-\hat{c}^{t} = j\Delta t\mathbf{k} \left\{ (\xi + 1-ac^2 -\xi) \left[ j\mathbf{k}' (\hat{g} + \kappa {k'}^2 \hat{c}) \right]_r^t \right\}_k
    \end{equation}
    which in turn writes:
    \begin{equation}
        (1+\xi \Delta t \kappa k^4)\hat{c}^{t+\Delta t} = (1+\xi \Delta t \kappa k^4)\hat{c}^{t} + j\Delta t \mathbf{k} \left\{ (1-ac^2) \left[ j\mathbf{k}' (\hat{g} + \kappa {k'}^2 \hat{c}) \right]_r^t \right\}_k
    \end{equation}
    for the considered mobility, the optimal value of $\xi$ is the one which satisfy the following condition: $\xi=\frac{1}{2}\left[\text{max}(M(c))+\text{min}(M(c))\right]$.
    The scheme is now slightly more computationally expensive than the explicit scheme but the constraint on the time discretization is greatly alleviated.