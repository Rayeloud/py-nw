\subsection{Benchmark problem}
    \subsubsection{Theoretical context}
    \begin{itemize}
        \item What is spinodal decomposition
        \item Why phase field modeling
    \end{itemize}
    \subsubsection{Results}
    \begin{itemize}
        \item Comparison of model results with Zhu et al's
        \item Model used as a benchmark to check solver consistency and stability
    \end{itemize}
\subsection{Consistency and stability assessment}
    Quantity to track: integral quantity --> total free energy
    \begin{figure}[H]
        \centering
        \includegraphics[width=0.7\textwidth]{example-image-a}
        \caption{placeholder}
        \label{fig:ftot}
    \end{figure}
    \subsubsection{Mesh refinement}
    \begin{figure}[H]
        \centering
        \includegraphics[width=0.7\textwidth]{example-image-a}
        \caption{placeholder}
        \label{fig:ftot_dx}
    \end{figure}
    \begin{itemize}
        \item Effect of the mesh refinement on the energy
        \item Effect of the mesh refinement on the numerical solution (error maps)
    \end{itemize}
    \subsubsection{Stability assessment}
    \begin{figure}[H]
        \centering
        \includegraphics[width=0.7\textwidth]{example-image-a}
        \caption{placeholder}
        \label{fig:ftot_dt}
    \end{figure}
    \begin{itemize}
        \item Von Neumann analysis --> stability assessment
        \item Integral quantity v.s. $dt$ with fixed mesh refinement
        \item Stability condition
    \end{itemize}
