\label{chap:2_4-nanowire}
In this section, the implemented model is used to reproduce some results presented in Roy et al.\ \cite{RoyVarmaGururajan2021} which studied surface-enhanced breakups in free-standing nanowires.
\subsection{Free standing nanowire model}
    The nanowire is approximated as a free-standing infinitely long cylinder of radius $R_1$. The physical domain is shown in \autoref{fig:2-free-standing-nw}. In addition, two configurations are considered, a single wire and a junction of two wires with an intersection angle of $\theta$. The geometry of the problem is made using the \texttt{gmsh} software\ \cite{geuzaine}.\\
    The order parameter $c$ in the phase-field model is defined as an indicator of wether a point belong to the nanowire, the interface or the vacuum,
    \begin{equation}
        c(\mathbf{x}, t) = \begin{cases}
            1 & \text{if } \mathbf{x}\ \text{in NW bulk} \\
            0.5 & \text{if } \mathbf{x}\ \text{on the interface} \\
            0 & \text{if } \mathbf{x}\ \text{in vacuum bulk}
        \end{cases}
    \end{equation}
    The contribution from the thermal annealing is modelled by adding a small stochastic perturbation to the order parameter $c$ in the initial condition \cite{}.
    \begin{equation}
        c(\mathbf{x}, 0) = \texttt{make\_composition\_field()} + c_{noise}(0.5-\texttt{RAND()})
    \end{equation}
    The Cahn-Hilliard equation solves the dynamics of a conserved order parameter $c$. Thus using this description, mass conservation is ensured. In addition, the total free energy is minimized when the system reaches equilibrium. The dynamics implemented being surface-driven, the minimization of the total free energy leads to the minimization of the surface energy.\\
    The phase field approach thus provide the same assumptions as McCallum and Plateau-Rayleigh albeit in a more general framework.
    \begin{figure}[H]
        \centering
        \includegraphics[width=0.8\textwidth]{chap2/2-free-standing-nw.png}
        \caption{Geometry of the (a) physical domain (b) junction with $\theta = \pi/2$ (c) single wire configuration. Visualization performed in \texttt{gmsh}.}
        \label{fig:2-free-standing-nw}
    \end{figure}
\subsection{Voxelisation}
    As previously mentioned in \autoref{chap:2_2-numerical_method}, the physical domain must be discretized as a regular grid. However, the geometry of the problem being complex, a voxelisation algorithm is used to convert the defined geometry into a regular grid. The voxelisation is performed by defining a coarse unregular mesh in \texttt{gmsh} and using octree search to find elements which are inside the volume of the nanowire. The voxelised nanowire is shown in \autoref{fig:2-voxels}.
    \begin{figure}[H]
        \centering
        \includegraphics[width=0.7\textwidth]{chap2/2-voxel.png}
        \caption{Voxelisation of the single nanowire geometry. (a) Side view (b) Cross-section of the voxelised geometry compared to the actual geometry (dark grey). Visualization performed in \texttt{Paraview}.}
        \label{fig:2-voxels}
    \end{figure}
    One can see that the voxelisation is not perfect and close attention to the discretisation of the grid is required. This argument is all the more crucial when the geometry of the nanowire is more complex, which is further detailed in \autoref{chap:3-results}.
\subsection{Results and comparison}
    Before presenting the main results, reproduction of the results presented in Roy et al.\ \cite{RoyVarmaGururajan2021} is performed on the two configurations i.e.\ single wire and junction.
    The simulation parameters used are the following.
    \begin{table}[H]
        \centering
        \begin{tabular}{|c|c|}
            \hline
            Parameter & Value \\
            \hline
            $\kappa$ & $1.0$ \\
            $A$ & $1.0$ \\
            $M_0$ & $0.5$ \\
            $\alpha$ & $0.5$ \\
            $c_{noise}$ & $10^{-3}$\\
            $\Delta x$ & $0.5$ \\
            $\Delta t$ & $1.0$ \\
            \hline
        \end{tabular}
        \caption{Simulation parameters used for the results}
        \label{tab:2-parameters}
    \end{table}
    \subsubsection{Single wire}
    The breakups of both infinitely long and finite single free-standing nanowires are respectively shown in \autoref{fig:2-single-wire}. The results of the breakup are in good agreement with Roy's. 
    \begin{figure}[H]
        \centering
        \includegraphics[width=\textwidth]{chap2/2-single-nw.png}
        \caption{Single free-standing nanowire simulation from (A) the implemented model (B) Roy et al. at (a) $t=500$ (b) $t=1500$ (c) $t=2700$ (d) $t=2800$.\ \cite{RoyVarmaGururajan2021}.}
        \label{fig:2-single-wire}
    \end{figure}
    \newpage
    Indeed, the free ends of the finite nanowire retract and bulge out. As the ends retracts, necking starts to form which then lead to the breakup of the nanowire into a series of spherical chunks. In addition, the spherical chunks are not of the same size, due to the coarsening of the wire. In the case of the infinitely long nanowire, the wire starts by coarsening until the instabilities grow sufficiently to induce necking and subsequent breakup of the nanowire.
    \subsubsection{Junction}
    The breakups in the junction configuration is presented in \autoref{fig:2-junction}. The observed evolution in good agreement with Roy's. 
    \begin{figure}[H]
        \centering
        \includegraphics[width=\textwidth]{chap2/2-junction.png}
        \caption{Simulation (A) from the implemented model (B) Roy et al. of the morphological evolution of two intersecting nanowires ($90^\circ$) at (a) $t=1000$ (b) $t=1750$ (c) $t=2500$ (d) $t=2750$ (e) $t=2800$ (f) $t=2900$.\ \cite{RoyVarmaGururajan2021}.}
        \label{fig:2-junction}
    \end{figure}
    The proximity between the two nanowires lead to local sintering. In addition, one can see in \autoref{fig:2-junction-zoom} that necking occurs at the location of the junction on each nanowire. As the wires coarsen and the instabilities grow, further necking is observed at the junction which lead to the breakup of the junction, leaving behind a spherical chunk. Afterwards, the free ends of the broken junction retracts and bulge out, leading to the formation of spherical chunks.
    \begin{figure}[H]
        \centering
        \includegraphics[width=0.75\textwidth]{chap2/2-zoom-junction.png}
        \caption{Zoom on the junction of the two nanowires at (a) $t=500$ (b) $t=1000$ (c) $t=1750$ from the implemented model.}
        \label{fig:2-junction-zoom}
    \end{figure}
