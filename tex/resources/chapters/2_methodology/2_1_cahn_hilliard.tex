\subsection{Formulation}
    The Cahn-Hilliard equation is a fourth-order partial differential equation which describes the evolution of a conserved order parameter $c$. The equation writes as follows,
    \begin{equation}\label{eq:2-ch}
        \frac{\partial c}{\partial t} = -\nabla \mathbf{J} = -\nabla \cdot \left( -M \nabla \mu \right)
    \end{equation}
    where $\mathbf{J}$ is the previously defined flux, $M$ is the mobility and $\mu$ is the generalized chemical potential defined as the variational derivative of the total free energy functional $\mathcal{F}$,
    \begin{equation}\
        \mu = \frac{\delta \mathcal{F}}{\delta c} = \frac{\partial f_0}{\partial c} - \kappa \Delta^2 c = g(c) - \kappa \Delta^2 c
    \end{equation}
    where $\mathcal{F}$ is the Ginzburg-Landau free energy functional describing isotropic binary systems, $f_0$ is the bulk free energy density, $g$ its derivative with respect to $c$ and $\kappa$ is the gradient energy coefficient.\\
    The Ginzburg-Landau free energy functional is defined as,
    \begin{equation}\label{eq:2-free-energy}
        \mathcal{F}(c) = \int_\Omega f_0(c) + \frac{\kappa}{2} |\nabla c|^2 d\Omega
    \end{equation}
\subsection{Variable Mobility}
    Usually, the mobility $M$ is considered constant and uniform across the domain which describes bulk-driven phase separation. However, as it was shown in the phase field formulation, the mobility is a function of the composition i.e.\ the order parameter $c$. Thus, making the mobility dependent on the order parameter leads to changes in the phase separation dynamics.
    Most commonly used mobility functions are quadratic or quartic polynomials of the order parameter $c$ which penalize the mobility in the bulk to enhance the mobility at the interface thus leading to surface-driven phase separation.\\
    The mobility function used in this study was introduced by Roy et al.\ \cite{RoyVarmaGururajan2021} and is defined as,
    \begin{equation}
        M(c) = 2M_0 \sqrt{|c - c^2|}
    \end{equation}
    where $M_0$ is the maximum mobility and $c$ is the order parameter.
    \begin{figure}[H]
        \centering
        \includegraphics[trim={0cm 0cm 1cm 1cm}, clip, width=.49\linewidth]{chap2/2-vm}
        \caption{Mobility function $M(c)$.}
        \label{fig:2-mobility}
    \end{figure}
    The above figure shows the variable mobility with respect to the order parameter $c$. The mobility is maximum at the interface and minimum in the bulk thus correctly describing surface-driven phase separation.
\subsection{Surface Diffusion in Silver Nanowires}
    Atomic diffusion is a key parameter when dealing with nanostructural evolution. In the case of metallic nanowires, surface diffusion play a major role in the morphological instability compared to bulk and grain diffusion. This can be understood by inspecting the Arrhenius plot of Silver self-diffusion.
    \begin{figure}[H]
        \centering
        \includegraphics[width=.49\linewidth]{chap2/2-arrhenius}
        \caption{Arrhenius plot of Silver self-diffusion. \cite{Wejranowski2020}}
        \label{fig:2-arrhenius}
    \end{figure}
    The latter shows the diffusivity exponential dependence to temperature. In addition, at temperature where the morphological instability in AgMNWs is observed i.e.\ $T=600\,K$, the surface self-diffusion is the dominant mechanism. Indeed, one can see that there are several order of magnitude between surface, grain boundary and bulk diffusion.\\
    Since the mobility in the sense of Cahn-Hilliard is related to the diffusivity,
    \begin{equation}
        \left[M\right] = \frac{L^2}{T\times E} = \frac{\left[D\right]}{E}
    \end{equation}
    it is, in a first approximation, correct to only consider surface mobility when modeling metallic nanowires.