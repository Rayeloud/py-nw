\section{Phase field modeling of a binary mixture}
        \begin{itemize}
            \item Definition of the conserved order parameters
            \item Description of the physics in the free energy
        \end{itemize}
\section{Numerical method}
    \subsection{Fourier spectral method}
        \subsubsection{Solving PDEs}
        \begin{itemize}
            \item Strengths and weaknesses (i.e. easy to implement, easy to parallelize, PDE --> $\{\}$ of ODEs, Periodic boundary condition "only", regular mesh only)
        \end{itemize}
        \subsubsection{Application to Cahn-Hilliard}
        \begin{itemize}
            \item Express the Cahn-Hiliard equation in Fourier space
            \item Rewrite to facilitate implementation
        \end{itemize}
    \subsection{Time marching scheme}
        \subsubsection{Forward Euler Method}
        \begin{itemize}
            \item Recall on the method
        \end{itemize}
        \subsubsection{Von-Neumann Analysis}
        \begin{itemize}
            \item Relation between time discretisation and spatial discretisation
            \item Highlight constraint
        \end{itemize}
        \subsubsection{Semi-implicit treatment}
        \begin{itemize}
            \item How to perform the semi-implicit treatment (cite Zhu et al.)
            \item Why is it useful
        \end{itemize}
    \subsection{Implementation}
        \subsubsection{}
        \begin{figure}[H]
            \centering
            \includegraphics[width=0.5\linewidth]{resources/pdf/chap2/implementation.png}
            \caption{Computation flowchart}
            \label{fig:enter-label}
        \end{figure}
\section{Numerical experiments}
    \subsection{Spinodal decomposition}
        \subsubsection{Theoretical context}
        \begin{itemize}
            \item What is spinodal decomposition
            \item Why phase field modeling
        \end{itemize}
        \subsubsection{Results}
        \begin{itemize}
            \item Comparison of model results with Zhu et al's
            \item Model used as a benchmark to check solver consistency and stability
        \end{itemize}
    \subsection{Consistency and stability assessment}
        Quantity to track: integral quantity --> total free energy
        \subsubsection{Mesh refinement}
        \begin{itemize}
            \item Effect of the mesh refinement on the energy
            \item Effect of the mesh refinement on the numerical solution (error maps)
        \end{itemize}
        \subsubsection{Stability assessment}
        \begin{itemize}
            \item Von Neumann analysis --> stability assessment
            \item Integral quantity v.s. $dt$ with fixed mesh refinement
            \item Stability condition
        \end{itemize}
\section{Nanowire morphological instability}
    \subsection{Model description}
        \subsubsection{Free standing nanowire}
        \begin{itemize}
            \item Nanowire model (from pentagonal base to circular base)
            \item Physical domain
        \end{itemize}
        \subsubsection{Phase field definition}
        \begin{itemize}
            \item Description of the NW-Vacuum Binary mixture
            \item Motivation for the description
        \end{itemize}
        \subsubsection{Discretisation}
        \begin{itemize}
            \item Regular mesh from the geometry --> Voxelisation
        \end{itemize}
    \subsection{Results}
        Table with the physical parameters used for the results
        \subsubsection{Single wire}
        \begin{itemize}
            \item $R^4$ v.s. $t_{bd}$
            \item $\lambda = C \times R_0$ (Mc Callum / Plateau Rayleigh)
            \item comparison with Roy et al.
        \end{itemize}
        \subsubsection{Junction}
        \begin{itemize}
            \item $\alpha$ v.s. $t_{bd}$
            \item mean curvature v.s. $t$
            \item comparison with Roy et al.
        \end{itemize}
    