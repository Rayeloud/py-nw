This chapter aims at introducing the methodology used throughout this work, i.e.\ the phase-field model utilized as well as its specifics. First, a section is dedicated to the employed Variable Mobility Cahn-Hilliard equation. Then, the numerical method used to implement the model is detailed. Subsequently, numerical experiments are performed using the implemented model in order to asses its consistency and stability. The model solutions are then compared to peer-reviewed benchmark problems \cite{JokisaariVoorheesGuyerWarrenHeinonen2017} in order to validate the model. Finally, some typical results concerning surface-enhanced breakups in free-standing nanowires are presented as a direct follow-up to Roy et al.'s work \cite{RoyVarmaGururajan2021} which will be further extended in \autoref{chap:3-results} and \autoref{chap:4-substrate}.
\section{Phase field modeling of a binary mixture}
    \subsection{Formulation}
    The Cahn-Hilliard equation, derived in \autoref{chap:1_3-phase-field}, is a fourth-order partial differential equation which describes the evolution of a conserved order parameter $c$. The equation writes as,
    \begin{equation}\label{eq:2-ch}
        \frac{\partial c}{\partial t} = -\nabla \cdot \mathbf{J} = -\nabla \cdot \left( -M \nabla \mu \right)\ ,
    \end{equation}
    where $\mathbf{J}$ is the previously defined diffusion flux in \autoref{eq:1-flux}, $M$ is the mobility function and $\mu$ is the generalized chemical potential defined as the variational derivative of the total free energy functional $\mathcal{F}$,
    \begin{equation}\label{eq:2-mu}
        \mu = \frac{\delta \mathcal{F}}{\delta c} = \frac{\partial f_0}{\partial c} - \kappa \Delta^2 c = wg'(c) - \kappa \Delta^2 c
    \end{equation}
    where $\mathcal{F}$ is the Ginzburg-Landau free energy functional describing isotropic binary systems, $f_0$ is the bulk free energy density, $wg'(c)$ its derivative with respect to $c$, $w$ the height of the double well potential and $\kappa$ is the gradient energy coefficient.\\
    The Ginzburg-Landau free energy functional is defined as,
    \begin{equation}\label{eq:2-free-energy}
        \mathcal{F}(c) = \int_\Omega f_0(c) + \frac{\kappa}{2} |\nabla c|^2 d\Omega\ .
    \end{equation}
\subsection{Variable Mobility}
    Usually, the mobility $M$ is considered constant and uniform across the domain which describes bulk-driven phase separation \cite{ZhuChenShenTikare1999,Voorhees2018}. However, as it was shown in the phase field formulation in \autoref{chap:1_3-phase-field}, the mobility can be a function of the composition i.e.\ the order parameter $c$ \cite{MoelansBlanpainWollants2008}. Thus, making the mobility dependent on the order parameter leads to changes in the phase separation dynamics.
    Most commonly used mobility functions are quadratic or quartic polynomials of the order parameter $c$ which penalize the mobility in the bulk to enhance the mobility at the interface, thus leading to surface enhanced phase separation \cite{ZhuChenShenTikare1999,Cahn1961,PesceMunch2021,CahnTaylor1994}.\\
    The mobility function used in this study was introduced by Roy et al.\ \cite{RoyVarmaGururajan2021} and is defined as,
    \begin{equation}
        M(c) = 2M_0 \sqrt{|c - c^2|}\ ,\label{eq:2-vm}
    \end{equation}
    where $M_0$ is the maximum mobility and $c$ is the order parameter. It is an adaptation of the classical mobility function $M(c) = |1-c^2|$ proposed by Langer et al.~\cite{Langer1975} for order parameters $c\in[-1,1]$. Both functions are shown in \autoref{fig:2-mobility}.
    \begin{figure}[H]
        \centering
        \includegraphics[trim={0cm 0cm 1cm 1cm}, clip, width=.49\linewidth]{chap2/2-vm}
        \caption{Comparison of the mobility function proposed by Roy et al.~\cite{RoyVarmaGururajan2021} and the classical mobility function proposed by Langer et al \cite{Langer1975}. The former has a smoother transition between the interfacial mobility and the bulk mobility. This smoother transition effectively reduces the stiffness of the equation, i.e. it improves its overall numerical stability \cite{Roy2021,LiTang2020}.}
        \label{fig:2-mobility}
    \end{figure}
    The above figure shows the variable mobility with respect to the order parameter $c$. The mobility is maximum at the interface defined as $c \approx 0.5$ and minimum in the bulk defined as $c \approx 0$ and $c \approx 1$. This leads to surface-driven dynamics in the system.
    The mobility function can be found in different forms in the literature but they all can be rewritten as,
    \begin{equation}
        M(c) = 4^n M_0 |c-c^2|^n\ .
        \label{eq:2-mobility-general}
    \end{equation}
\subsection{Surface Diffusion in Silver Nanowires}
    Atomic diffusion is a key parameter when dealing with nanostructural evolution. In the case of metallic nanowires, Rhead \cite{Rhead1963} showed through experimental work that surface diffusion is the dominant mass transport mechanism compared to bulk and grain diffusion. It was showed that self-diffusion coefficients fit an Arrhenius law, i.e. $D_s = D_0 \exp(Q_s/kT)$, with $D_s$ the self-diffusion coefficient. \autoref{fig:2-arrhenius} shows the Arrhenius plot of Silver surface, grain and bulk self-diffusion.
    \begin{figure}[H]
        \centering
        \includegraphics[width=.49\linewidth]{chap2/2-arrhenius}
        \caption{Arrhenius plot of Silver self-diffusion as a function of temperature. \cite{Wejrzanowski2017}}
        \label{fig:2-arrhenius}
    \end{figure}
    In addition, at temperature where the morphological instability in AgMNWs is observed i.e.\ $1/T\approx 1.6 \,K^{-1}$ or $T=600\,K$ \cite{Langley2014}, the surface self-diffusion is the dominant mechanism. Indeed, one can see that several orders of magnitude separate surface self-diffusion from grain boundary and bulk self-diffusion.\\
    The mobility function $M(c)$ is related to the interdiffusion or self-diffusion coefficient $D$ as shown by Moelans et al.~\cite{MoelansBlanpainWollants2008}. In general, the mobility is defined as,
    \begin{equation}
        M(c) = \frac{D}{\partial^2 f_0 / \partial c^2}\ .
    \end{equation}
    In a simplified framework, modeling the surface self-diffusion as the dominant mechanism can be achieved by considering the mobility function as a function of $c$ as shown in \autoref{eq:2-vm}.
\section{Numerical method}\label{chap:2_2-numerical_method}
    Different numerical strategies can be employed for phase-field modeling \cite{pfhub}. Here is a brief overview of different spatial discretization methods used in phase-field modeling:
\begin{itemize}
    \item Finite difference method
    \item Finite element method
    \item Finite volume method
    \item Fourier spectral method
\end{itemize}
Each methods has its own strength and weaknesses. In this study, the Fourier spectral method is used to solve the phase-field model considered. This method is fairly simple to implement and offer a high spatial error convergence rate. However, due to the mathematical properties of the Fourier transform, the method is limited to periodic boundary conditions and to the use of regular grids for spatial discretization. Thus, certain strategies are employed to overcome these limitations, especially the use of voxelization to represent any arbitrary shape in a regular grid.
\subsection{Fourier spectral method}
    Recall \autoref{eq:2-ch}, expressed as,
    \begin{equation}
        \frac{\partial c}{\partial t} = \nabla \cdot \left( M \nabla \mu \right)
    \end{equation}
    The solution to this equation with the Fourier spectral method starts by taking the forward Fourier transform of both sides of the equation.
    \begin{align}
        \frac{\partial \hat{c}}{\partial t} &= \Bigl\{\nabla \cdot ( M(c) \nabla \mu ) \Bigl\}_k\\
        \implies \frac{\partial \hat{c}}{\partial t} &= j\mathbf{k} \cdot \Bigl\{ M(c) \left[j\mathbf{k} \cdot \right(\hat{g(c)} + \kappa k^2 \hat{c}\left)\right]_r \Bigl\}_k
    \end{align}
    where $\hat{c}$ is the Fourier transform of $c$, $\{\cdot\}_k$ denotes the Forward Fourier transform of the expression inside the brackets, $\left[\cdot\right]_r$  denotes the Inverse Fourier transform of the expression in the brackets, $j$ is the pure imaginary number, $\mathbf{k}$ and $k$ are the wave vector and the wave vector magnitude respectively.\\
    Thus the problem simplifies to solving a system of ODEs in the Fourier reciprocal space.
\subsection{Time marching scheme}
    A time marching scheme is now needed to solve the system of ODEs in the Fourier reciprocal space. The simplest method is the Forward Euler method which is a first-order accurate method.
    \begin{equation}
        \hat{c}^{n+1} = \hat{c}^n + \Delta t j\mathbf{k} \cdot \Bigl\{ M(c) \left[j\mathbf{k} \cdot \left(\hat{g(c^n)} + \kappa k^2 \hat{c}^n\right)\right]_r \Bigl\}_k
    \end{equation}
    where $\Delta t$ is the time step and $n$ is the time step index.
    Unfortunately, the Forward Euler method is conditionally stable and suffer from a severe time step constraint. Indeed, by performing a Von-Neumann analysis on the above scheme, the time constraint can be derived.
    \subsubsection{Von-Neumann analysis}
    The model is most limited when the mobility function is at its maximum $\texttt{max}(M(c)) = M_0$
    \begin{align*}
        \hat{c}^{n+1} &= \hat{c}^n + \Delta t j\mathbf{k} \cdot \Bigl\{ M_0 \left[j\mathbf{k} \cdot \right(\hat{g(c^n)} + \kappa k^2 \hat{c}^n\left)\right]_r \Bigl\}_k\\
        &= \hat{c}^n - \Delta t M_0 k^2 \left(\hat{g(c^n)} + \kappa k^2 \hat{c}^n \right)
    \end{align*}
    Considering the most limiting factor i.e. the fourth order term, the derivative of the bulk free energy density in the Fourier reciprocal space is considered negligible. Thus, one writes,
    \begin{equation*}
        \hat{c}^{n+1} = \left(1 - \Delta t M_0 \kappa k^4 \right) \hat{c}^n
    \end{equation*}
    This finally leads to the amplification factor,
    \begin{align*}
        |\lambda| = |1 - \Delta t M_0 \kappa k^4| &\leq 1\\
        \implies \Delta t M_0 \kappa k^4 &\leq 1
    \end{align*}
    Which leads to the following Courant Friedrich Lewy (CFL) condition,
    \begin{equation}\label{eq:2-cfl}
        \Delta t = \texttt{CFL} \frac{1}{M_0 \kappa k^4} = \texttt{CFL} \frac{dx^4}{M_0 \kappa}
    \end{equation}
    \subsubsection{Semi-implicit treatment}
    To circumvent this constraint, a semi-implicit treatment inspired by Zhu et al.\ \cite{ZhuChenShenTikare1999} is performed by splitting the mobility into two parts i.e.\ $M(c) \rightarrow (M(c) - \alpha), \alpha$, with $\alpha$ the stabilization factor. This leads to a semi-implicit treatment of the fourth order term.
    The scheme is then written as,
    \begin{equation}
        \begin{aligned}
            \frac{\hat{c}^{n+1}-\hat{c}^n}{\Delta t} = j\mathbf{k} \cdot \Bigl\{ (M(c) - \alpha) \left[j\mathbf{k} \cdot \left(\hat{g(c^n)} + \kappa k^2 \hat{c}^n\right)\right]_r \Bigl\}_k\\
            + \alpha j\mathbf{k} \cdot \left[j\mathbf{k} \cdot \left(\hat{g(c^n)} + \kappa k^2 \hat{c}^{n+1}\right)\right]
        \end{aligned}
    \end{equation}
    Expanding the above equation and isolating both $\hat{c}^{n+1}$ and $\hat{c}^n$ leads to the following semi-implicit Fourier Spectral scheme,
    \begin{equation}
        \begin{aligned}
            \hat{c}^{n+1} = \hat{c}^n + \frac{\Delta t j\mathbf{k} \cdot \Bigl\{ M(c) \left[j\mathbf{k} \cdot \left(\hat{g(c^n)} + \kappa k^2 \hat{c}^n\right)\right]_r \Bigl\}_k}{1 + \alpha \Delta t \kappa k^4}
        \end{aligned}
    \end{equation}
    Zhu et al.\ \cite{ZhuChenShenTikare1999} showed that choosing $\alpha = \frac{1}{2} (\texttt{max}(M(c)) + \texttt{min}(M(c)))$ alleviates the time step constraint of the explicit scheme.

\subsection{Diffuse filtering scheme}
    In addition to the semi-implicit treatment, a diffuse filtering scheme inspired by Sinhababu et al.\ \cite{SinhababuBhattacharya2022} is employed to further stabilize the scheme. Indeed, the Fourier spectral method is know to suffer from the Gibbs phenomenon\ \cite{vSircaHorvat2012}--\cite{Fornberg1996} which leads to numerical oscillations near interfaces. Since the considered problems are interfacial physics problems, numerical oscillations are detrimental to the solution. Thus a dealiasing method is employed to filter out high frequency Fourier modes.
    The diffuse filtering scheme filters out Fourier modes of frequency higher than $|k| \geq \frac{\sqrt{2}N}{3}$ with $N$ the lowest dimension of the grid. Following the notation in \cite{SinhababuBhattacharya2022}, it is defined as follows,
    \begin{equation}
        \mathcal{W}(\mathbf{k}) = \frac{1}{2}\left[1+\tanh{\left(\frac{\frac{\sqrt{2}N}{3} - \sqrt{k_x^2 + k_y^2 + k_z^2}}{\epsilon}\right)}\right]
    \end{equation}
    with $\epsilon = 3\times\texttt{max}(\Delta k_x, \Delta k_y, \Delta k_z)$, with $\Delta k_i$ the grid spacing in the $i$ direction in the Fourier reciprocal space.\\
    The filtered Fourier transform of an arbitrary function now writes $\{f\}_k\rightarrow \hat{f} \cdot \mathcal{W}(\mathbf{k})$. The filtered reciprocal space is represented in \autoref{chap2:2-filtered-fourier}.
    \begin{figure}
        \centering
        \includegraphics[width=0.5\linewidth]{chap2/2-diffuse-filter}
        \caption{Filtered Fourier reciprocal space.}
        \label{chap2:2-filtered-fourier}
    \end{figure}
    \subsection{Implementation}
    The numerical scheme implementation is detailed in \autoref{fig:2-flowchart} and is inspired from \cite{Roy2021}. The implementation is done in \texttt{Python3.8} using the \texttt{PyTorch} library\ \cite{Ansel2024}.
    \begin{figure}[H]
        \centering
        \includegraphics[width=0.8\linewidth]{chap2/2-implementation.png}
        \caption{Flowchart of the numerical scheme implementation.}
        \label{fig:2-flowchart}
    \end{figure}
    \texttt{PyTorch} is a high-performance deep learning library which offers automatic differentiation and and more importantly GPU acceleration which is crucial for solving phase-field problems. The Fast Fourier Transforms (resp. Inverse Fast Fourier Transforms) are computed using \texttt{torch.fft.rfftn} (resp. \texttt{torch.fft.irfftn}) since the phase-field is real valued. In addition, the \texttt{gmsh} python library is used to generate geometries, i.e.\ the initial configuration of the phase field which are then voxelize to fit the required regular grid. The details on voxelization are provided in \autoref{chap:2_4-nanowire}. Finally, the visualisation is perfomed using \texttt{Paraview}.\\
    Numerical experiments on the implemented model are now performed on a \texttt{MacBook Pro 16'' M1 Max} with $64\,\text{GB}$ of RAM.

\section{Numerical experiments}\label{chap:2_3-numerical_experiments}
    \subsection{Benchmark problem}
    With the growing interest in the field of phase field modeling, Jokisaari et al.\ \cite{JokisaariVoorheesGuyerWarrenHeinonen2017} put together a series of benchmark problems to assess the accuracy and efficiency of newly implemented phase field solvers. The considered benchmark problem is the spinodal decomposition of a binary mixture which is a standard problem in the field of phase field modeling. Spinodal decomposition might be one of the simplest problem to model, but it is also highly relevant as the simulated physics are the basics of more challenging problem such as nanowire morphological instability.
    \subsubsection{Problem 1 statement}
    The free energy of the system is defined as \autoref{eq:2-free-energy} where, in this case, the bulk free energy density $f_0$ is defined as,
    \begin{equation}
        f_0(c) = A (c-c_{\alpha})^2 (c_{\beta}-c)^2
    \end{equation}
    with $c_{\alpha}$ and $c_{\beta}$ the composition in the bulk of the binary mixture $\alpha$--$\beta$.\\
    The mobility is considered constant and uniform across the domain.
    Finally, the considered model parameters are the following,
    \begin{equation}
        A = 5 \quad \kappa = 2 \quad M = 5 \quad c_{\alpha} = 0.3 \quad c_{\beta} = 0.7 \quad c_0 = 0.5 \quad \epsilon = 0.01
    \end{equation}
    The computational domain is defined as a square box of side length $L=200\,(-)$ with periodic boundary conditions on all sides.\\
    The initial condition is defined as follows,
    \begin{equation}
        \begin{aligned}
            c(x, y, 0) = c_0 + \epsilon [\cos{(0.105x)}\cos{(0.11y)}+\left[\cos{(0.13x)}\cos{(0.087x)}\right]^2\\
            +\cos{(0.025x-0.15y)}\cos{(0.07x-0.02y)}]
        \end{aligned}
    \end{equation}
    \begin{figure}[H]
        \centering
        \includegraphics[width=0.5\textwidth]{chap2/2-jokisaari-problem.png}
        \caption{Computational domain and initial condition of the benchmark problem.}
        \label{fig:2-jokisaari-problem}
    \end{figure}
    \subsubsection{Problem 2 statement}
    In addition to this benchmark problem, the implemented model is compared to the results of Zhu et al.\ \cite{ZhuChenShenTikare1999} which studied coarsening kinetics using a variable mobility Cahn-Hilliard equation. Snapshots of the microstructural evolution are compared to the one presented in\ \cite{ZhuChenShenTikare1999}.\\
    The computational domain is defined as a square box of side length $L=1024\,(-)$ with periodic boundary conditions on all sides. The initial condition is defined as follows,
    \begin{equation}
        c(x, y, 0) = c_0 + \epsilon \left[ 0.5 - \texttt{RAND()} \right]
    \end{equation}
    with $c_0$ the critical composition of the binary mixture and $\texttt{RAND()}$ a random number generator ($X\sim \mathcal{U}(0, 1)$).\\
    Zhu et al.\ relied on another definition of the mobility function, $M(c)=|1-c^2|$ and a scaled order parameter $c\in\left[-1,1\right]$ whereas the implemented model defines the order parameter as $c\in\left[0,1\right]$. However, the goal is to compare the microstructural evolution under surface-driven phase separation conditions and not exact quantitative comparison.
\subsection{Consistency and stability assessment}
    The phase-field model now implemented, the next step consists in checking the consistency and stability of the model. The consistency is ensured if the numerical solution converges to the real solution as the regular mesh is refined. As for the stability, using the CFL condition previously established in \autoref{eq:2-cfl}, one can empirically assess wether further temporal refinement is needed.\\
    The total free energy of the system is used as a metric to assess both the consistency and stability of the model as it is an integral quantity which is best suited for this purpose. It is compared to the one reported by Jokisaari et al.\ \cite{JokisaariVoorheesGuyerWarrenHeinonen2017} for the benchmark problem.
    \subsubsection{Mesh refinement}
    The following figure shows the effect of the mesh refinement on the total free energy with respect to time.
    \begin{figure}[H]
        \centering
        \includegraphics[width=0.7\textwidth]{chap2/2-consistency.pdf}
        \caption{Total free energy with respect to time for different $\Delta x$ with $\Delta t = 0.5$.}
        \label{fig:ftot_dx}
    \end{figure}
    One can see that as the mesh is refined, the total free energy converges to the one reported by Jokisaari et al.\ \cite{JokisaariVoorheesGuyerWarrenHeinonen2017}. The slight discrepancies are mainly due to time marching scheme and the spatial discretization used in the implemented model. Jokisaari et al.\ used a time adaptive scheme as well as an adaptive mesh refinement. But for the purpose of this work, a fixed time step and uniform regular grid provide sufficient accuracy.\\
    The mean square error $MSE$ is also computed over the simulated time and is reported in the following table.
    \begin{table}[H]
        \centering
        \begin{tabular}{|c|c|}
            \hline
            $\Delta x$ & $MSE$ \\
            \hline
            $5.0$ & $59.5024$ \\
            $2.0$ & $6.24622$ \\
            $1.0$ & $6.18410$ \\
            $0.5$ & $6.14669$ \\
            \hline
        \end{tabular}
        \caption{Mean square error of the total free energy with respect to the mesh refinement.}
        \label{tab:ftot_mse}
    \end{table}
    As the mesh becomes finer, the mean square error decreases less and less.\\
    In addition, for coarse mesh, the total free energy highly deviates from the expected trend. This is mainly due to the poor resolution of the interface. Indeed, to correctly capture the phase separation, the spatial discretization must be fine enough to resolve the interface between the two phase.
    \subsubsection{Stability assessment}
    The following figure shows the effect of temporal refinement at constant spatial discretization
    of the total free energy with respect to time.
    \begin{figure}[H]
        \centering
        \includegraphics[width=0.7\textwidth]{chap2/2-stability.pdf}
        \caption{Total free energy with respect to time for different $\Delta t$ with $\Delta x=1.0$.}
        \label{fig:ftot_dt}
    \end{figure}
    One can see that the scheme is unstable for $\Delta t \approx 2.0$ and the total free energy is subject to a sudden increase which is not physical. In addition, one can see that further temporal refinement does not significantly lead to a better precision of the solution and comes with a large computational cost.

    \subsection{Validation}
    The consistency and stability of the model now established, the model can now be validated against another benchmark problem, i.e.\ Zhu et al.\ coarsening kinetics problem. Since the authors only provided snapshots of the microstructural evolution, the evolution of the total free energy is not reported. In addition, error maps are not reported since the initial condition relies on a random number generator. Thus, only the microstructural evolution behavior is assessed.
    The simulation parameters are the following,
    \begin{equation}
        A = 1 \quad \kappa = 1 \quad M_0 = 0.5 \quad \alpha = 0.25 \quad \Delta x = 1.0 \quad \Delta t = 1.0
    \end{equation}
    \begin{figure}[H]
        \centering
        \includegraphics[width=0.8\textwidth]{chap2/2-zhu-problem.png}
        \caption{Snapshots of the microstructural evolution of the coarsening kinetics problem. (a) implemented model, (b) Zhu et al.\ \cite{ZhuChenShenTikare1999}.}
        \label{fig:2-zhu-problem}
    \end{figure}
    One can see that the general in \autoref{fig:2-zhu-problem} behavior of surface-driven phase separation is captured by the implemented model. The microstructural evolution behavior is consistent with the one reported in Zhu et al.\ \cite{ZhuChenShenTikare1999}.

\section{Nanowire morphological instability}\label{chap:2_4-nanowire}
    \subsection{Free standing nanowire model}
\begin{figure}[H]
    \centering
    \includegraphics[width=0.7\textwidth]{example-image-a}
    \caption{placeholder}
    \label{fig:2-free-standing-nw}
\end{figure}
    \subsubsection{Free standing nanowire}
    \begin{itemize}
        \item Nanowire model (from pentagonal base to circular base)
        \item Physical domain
    \end{itemize}
    \subsubsection{Phase field definition}
    \begin{itemize}
        \item Description of the NW-Vacuum Binary mixture
        \item Motivation for the description
    \end{itemize}
\subsection{Voxelisation}
    \begin{figure}[H]
        \centering
        \includegraphics[width=0.7\textwidth]{example-image-a}
        \caption{placeholder}
        \label{fig:2-voxels}
    \end{figure}
    placeholder
\subsection{Results and comparison}
    Table with the physical parameters used for the results
    \begin{table}[H]
        \centering
        \begin{tabular}{|c|c|}
            \hline
            Parameter & Value \\
            \hline
            $R_1$ & $12 \Delta x$ \\
            $R_2$ & $12 \Delta x$ \\
            $\kappa_c$ & $1.0$ \\
            $A$ & $1.0$ \\
            $M$ & $1.0$ \\
            $\alpha$ & $0.5$ \\
            $\Delta x$ & $0.5$ \\
            $\Delta t$ & $1.0$ \\
            \hline
        \end{tabular}
        \caption{Similation parameters used for the results}
        \label{tab:2-parameters}
    \end{table}
    \subsubsection{Single wire}
    \begin{itemize}
        \item $R^4$ v.s. $t_{bd}$
        \item $\lambda = C \times R_0$ (Mc Callum / Plateau Rayleigh)
        \item comparison with Roy et al.
    \end{itemize}
    \subsubsection{Junction}
    \begin{itemize}
        \item $\alpha$ v.s. $t_{bd}$
        \item mean curvature v.s. $t$
        \item comparison with Roy et al.
    \end{itemize}

    