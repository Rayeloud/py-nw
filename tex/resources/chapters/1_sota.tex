% This first chapter introduces the required theoretical tools needed to study the morphological instability of metallic nanowire. The first section is dedicated to the introduction of the concept of transparent conducting materials/electrodes (TCM/Es) and more specially metallic nanowire-based TCEs. Metallic nanowire networks are then compared to transparent conducting oxides, the industry leading transparent conducting materials nowadays. Afterwards, a review of the different type of electrical activation methods are discussed with a focus on thermal annealing which shed light on the thermal instability of such metallic nanonetworks. The thermal instability leads to the morphological instability of the nanowires which is introduced in the second section. 
This first chapter introduces the theoretical tools needed to study the morphological instability of metallic nanowires numerically. To do so, three concepts are discussed.\\
First, the notion of metallic nanowires (MNWs) by introducing the concept of transparent conducting materials/electrodes, their field of applications and, in the case of MNWs, the different electrical activation methods used in practice, with a focus on thermal annealing. \\
In addition, the concept of morphological instability is discussed through the introduction of the thermal instability that metallic nanowires suffer, especially Ag-MNWs. This section covers the works of Plateau-Rayleigh in their first theoretical description of the phenomenon. \\
Finally, a detailed description of phase field modeling theory, the model utilised in this study, is performed.
\section{Metallic nanowires}
    \subsection{Transparent conducting materials}
    \subsubsection{Introduction}
        Transparent conducting materials constitute a special class of materials. Exhibiting both optical transparancy and electrical conductivity, their use skyrocketed since the 1950s with the technological advances made. 
    \subsubsection{Haacke Figure of Merit}
        placeholder
    \subsubsection{Transparent Conducting Oxides}
        placeholder
    \subsubsection{Metallic Nanowire networks}
        placeholder
\subsection{Electrical activation methods of Metallic Nanostructures}
    \subsubsection{Introduction}
        placeholder
    \subsubsection{Overview of different activation methods}
        placeholder
    \subsubsection{Thermal annealing}
        placeholder
\section{Morphological instability}
    In the literature, this phenomenon is often referred to as Plateau-Rayleigh instability, dewetting or spheroidization. This section focuses on the historical description of such instabilities and the discrepancies between the theoretical predictions and the experimental observations in metallic nanowires.
\subsection{Theoretical description}
        In 1873, Joseph Plateau conducted a series of experiments which led to the observations of instability in liquid jets \cite{Plateau1873}. He observed that a \textit{`vertically falling stream of water'} breaks up into a series of droplets if the length of the stream exceeds a multiple of the initial diameter of the stream. Later, in 1878, Lord Rayleigh provided a theoretical model to explain the phenomenon observed by Plateau\ \cite{Rayleigh1878}. His arguments were related on the minimization of surface energy. His model confirmed Plateau's observations and predicted that the jet would break up into droplets with a characteristic wavelength of $\lambda = 9.016\,R_0$, with $R_0$ the initial radius of the jet. This phenomenon is now known as Plateau-Rayleigh instability.
        \begin{figure}[H]
                \centering
                \includegraphics[width=0.7\textwidth]{chap1/1-plateau-rayleigh.png}
                \caption{Plateau-Rayleigh instability in a liquid jet. Adapted from\ \cite{RutlandJameson1971}}
                \label{fig:plateau-rayleigh}
        \end{figure}
        Following this description, Nichols and Mullins\ \cite{NicholsMullins1965} worked on the stability of solids of revolution. By considering surface-driven mass transport as the primary driving force, they developed a numerical method to compute \textit{`the kinetics of the shape changes of any solid of revolution'}. They found that the solution to a cylindrical rod is a series of equally-spaced spheres along the direction of the rod, the spheroidization of the rod.\\%their model was able to predict the spheroidization of cylindrical rods.
        McCallum et al.\ \cite{McCallumVoorheesMiksisDavisWong1996} extended this work for the case of cylindrical rods deposited on a substrate with varying contact angle. Their work showed that the presence of the substrate provides a stabilizing effect to the morphological instability phenomenon. Indeed, their work showed a decreasing trend of the non-dimensional growth rate $\sigma_m$ with respect to the contact angle $\alpha$ modeling the presence of a substrate. The function relating $\sigma_m$ to the contact angle $\alpha$ is represented in \autoref{fig:sigma_m_mccallum}.
        \begin{figure}[H]
                \centering
                \includegraphics[width=0.8\textwidth]{chap1/1-mccallum-sigma.png}
                \caption{Plot of (a) the maximum nondimensional growth rate $\sigma_m$ with respect to the contact angle $\alpha$ (b) the stability regions with respect to the contact angle $\alpha$. The solid line represents the value of $k_c^2$, the critical nondimensional wavenumber associated to a nondimensional growth rate $\sigma=0$. The dashed line represents the value of $k_m^2$, the nondimensional wavenumber associated to the maximum nondimensional growth rate $\sigma = \sigma_m$ \cite{McCallumVoorheesMiksisDavisWong1996}.}
                \label{fig:sigma_m_mccallum}
        \end{figure}
        The wavelength of the instability predicted by McCallum is of the same order of magnitude as the one predicted by Plateau-Rayleigh for free-standing jets, however for contact angle below $\theta=\pi$ corresponding to the free-standing case, the wavelength is slightly larger. Since it is inversely proportional to the wavenumber $k$, and the growth rate is smaller. Thus, the presence of the substrate is stabilizing.
\subsection{Application to metallic nanowires}
        In their experimental work, Langley et al.\ \cite{Langley2014} found that the distance between the nanoparticles after spheroidization is of the same order of magnitude as the one predicted by Plateau-Rayleigh but is slightly larger. Inspired by observations and by the work of McCallum, Balty et al.\ \cite{BaltyBaretSilhanekNguyen2024} showed that the predictions provided by McCallum are in better agreement with the measurements in an extended series of experiments summarized in \autoref{fig:1-plateau-mccallum}.
        \begin{figure}
                \centering
                \includegraphics[width=0.48\textwidth]{chap1/1-balty.png}
                \caption{Fitted mean instability wavelength with respect to the initial mean radii of AgNWs samples. The grey area represents the confidence interval of the fitted curve. McCallum theory provides a prediction closer to the experimental results than Plateau-Rayleigh's\ \cite{BaltyBaretSilhanekNguyen2024}.}
                \label{fig:1-plateau-mccallum}
        \end{figure}
        Thus, McCallum model provides a good theoretical framework to understand the morphological instability of metallic nanowires which paves the way towards stabilization strategies and a better understanding of the mechanism at play at that scale.\\
        However, the model is limited to the description of idealized infinitely long truncated cylinders, and cannot take into account the effect of a junction between two nanowires, the effect of the length/radius ratio, the effect of the crystalline nature of the nanowire which yields a pentagonal cross-section, and more.\\\\
        Numerical simulations are thus required to deepen the understanding of the phenomenon and to provide a more accurate and complete description of the morphological instability of metallic nanowires, which is the focus of this study. There exists several modeling theory one can utilize to model physics at the mesoscale, the scale encompassing the nanowire size, the scale in-between atomistic and macroscopic scales.\\
        Molecular Dynamics (MD) is a powerful numerical tool for simulating the motion of atoms in a molecular system. However, MD is computationally expensive and is limited to the study of small systems over small time scales with time steps on the order of the femtoseconds \cite{Tuckerman2000}. It is thus ill-suited for the study of the morphological instability and the resulting necking of nanowires.
        One of the most promising theory for modeling physical processes at the mesoscale, i.e. the scale in-between atomistic and macroscopic scales, is the Phase-field theory which gained popularity in the last decades for its versatility and ability to model complex geometries \cite{BartelsMosler2015}.

\section{Phase field modeling theory}
    \subsection{Theoretical context}
    In materials science, many important processes take place at the mesoscale. Mesoscale processes can impact on the measured macro-properties of a system\ \cite{MoelansBlanpainWollants2008}. Thus, accurate models which simulate the physics at that scale are required to better understand and predict the properties of physical systems.\\
    Initially, the sharp-interface approach was used to study the physics at that scale at the mesoscale. However, many processes at the this scale suffer from this mathematical definition and become almost intractable using this approach. Interface tracking with complex geometries (e.g. during dendritic growths) and topological changes (e.g. merging of two particles) are particularly challenging\ \cite{Emmerich2008}.\\
    Indeed, using a sharp-interface approach, consider the case of precipitate growth schematised in \autoref{fig:1-sharp-interface}.
    \begin{figure}[H]
        \centering
        \includegraphics[width=0.44\textwidth]{chap1/1-free-boundary.png}
        \caption{Sharp interface approach to 1D growth of a spherical particle of radius $R$. The figure represents the concentration profile along the radial direction. The concentration values $c^\beta$, $c^\alpha$ and $c_0$ are respectively the equilibrium concentration in the $\beta$ phase, $\alpha$ phase and the alloy. Adapted from \cite{Voorhees2018,MoelansBlanpainWollants2008}.}
        \label{fig:1-sharp-interface}
    \end{figure}
    The precipitate in the $\beta$ phase grows by diffusion. The concentration profile is solved using Fick's second law of diffusion \cite{Gottstein2004} in both phases and the interface position, i.e. the radius $R$, is determined by the equilibrium condition at the interface. This writes as follows,
    \begin{equation}
        \begin{aligned}
            &\frac{\partial c^i}{\partial t} = D^i \nabla^2 c^i\quad,\quad i=\alpha, \beta\\
            &\frac{\partial R}{\partial t} = D \frac{\partial c}{\partial r}\Bigl|_{r=R(t)}\\
            &\mu^\alpha(c^\alpha_{int}) = \mu^\beta(c^\beta_{int}).
        \end{aligned}
    \end{equation}
    The first equation, Fick's second law, is obtained by combining Fick's first law which states that the flux of particles is proportional to the gradient of concentration $\nabla c$ with the continuity equation \cite{Gottstein2004}. $D^i$ is the diffusion constant related to the phase $i$ and $c^i$ is the concentration of particles in the phase $i$.\\
    The second equation tracks the motion of the interface. Finally, the third equation describes the thermodynamic constraint which states that both phases are in equilibrium at the interface\ \cite{Voorhees2018, Gottstein2004,Emmerich2008, LeeHuhJeongShinYunKim2014,MoelansBlanpainWollants2008}.\\
    The issue with this approach is that both equations depend on the position of the interface which consequently depends on the composition in each phase. This kind of problem is referred to as a free boundary problem\ \cite{JokisaariVoorheesGuyerWarrenHeinonen2017}.\\\\
    A solution to this problem was introduced by Langer\ \cite{LangerBaronMiller1975}. He proposed a description using a single equation which holds true in the entire domain. To achieve this, the sharp interface is approximated by a finite width interface, a diffuse-interface, schematised in \autoref{fig:1-diffuse-interface}. This approach was first thought to be \textit{`too complex to be useful'} since it would require a fine enough mesh to render the diffuse interface\ \cite{Voorhees2018}. However, with the technological advances, diffuse-interface approach is becoming the standard when studying microstructural evolution, in particular phase-field modeling.
    \begin{figure}[H]
        \centering
        \includegraphics[width=0.42\textwidth]{chap1/1-diffuse-interface.png}
        \caption{Diffuse interface approach to 1D growth of a spherical particle of radius $R$. Adapted from \cite{Voorhees2018,MoelansBlanpainWollants2008}}
        \label{fig:1-diffuse-interface}
    \end{figure}
    In this context, the microstructure is described by a set of continuous fields which vary smoothly across the interface. Within each phase, the field has the same value and meaning as in the sharp interface approach. The position of the free interface can be retrieved through contours of constant values of the field variable. In addition, no constraint are required at the interface. For example, consider the case of a binary system of composition $c$ composed of an $\alpha$ and a $\beta$ phase which respectively have a composition $c=c_\alpha^{eq}$ and $c=c_\beta^{eq}$ at equilibrium. The regions of the system where the field variable $c(\mathbf{x}, t)=c_\alpha^{eq}$ and $c(\mathbf{x}, t)=c_\beta^{eq}$ corresponds respectively to the $\alpha$ and $\beta$ phase whereas the regions where $c(\mathbf{x},t)$ is between $c_\alpha^{eq}$ and $c_\beta^{eq}$ corresponds to the interface between both phases, as shown in \autoref{fig:1-phase-field} \cite{LeeHuhJeongShinYunKim2014}.
    \begin{figure}[H]
        \centering
        \includegraphics[width=0.3\textwidth]{chap1/1-phase-field.png}
        \caption{Example of a two phase microstructure with a phase field variable $c$ in 2D. \cite{LeeHuhJeongShinYunKim2014}}
        \label{fig:1-phase-field}
    \end{figure}
\subsection{Fundamental principles and Formulation}
    The microstructure is described by a set of continuous fields \cite{LeeHuhJeongShinYunKim2014,Voorhees2018,Cahn1961,Biner2017-1,CahnHilliard1958,Cahn1959}, i.e. the phase-field variables, field variables or order parameters. These variables can either be conserved or non-conserved depending on the phase-field model employed. Conserved order parameter often refers to the local composition whereas non-conserved order parameter often refers to crystal structure or to the phase of a composition (e.g. solid-liquid). Conserved fields variables satisfies a continuity equation which ensures conservation of the quantity whereas non-conserved field variables have no conservation constraint \cite{CahnHilliard1958,Cahn1959}.\\
    The driving force of microstructural dynamics is the minimization of the free energy of the system \cite{MoelansBlanpainWollants2008,Voorhees2018,CahnHilliard1958,Cahn1959} can be written as,
    \begin{equation}
        F = F_{bulk} + F_{int} + F_{source}\ ,
    \end{equation}
    with $F_{bulk}$ the free energy associated to the bulk of the system, $F_{int}$ the free energy associated to interfacial interactions and $F_{source}$ the free energy associated to additional sources of energy such as elastic strains, electromagnetic fields, and more \cite{Voorhees2018,MoelansBlanpainWollants2008}.\\
    Classically, thermodynamic properties are assumed homogeneous throughout the system. However, in the case of phase-field modeling, the system is considered `\textit{nonuniform}', i.e. `\textit{a system having a spatial variation in one of its intensive scalar properties, such as composition or density}' \cite{CahnHilliard1958}. The total free energy of a nonuniform system of volume $\Omega$ is given by,
    \begin{equation}
        \begin{aligned}
            \mathcal{F}(c) &= \int_{\Omega} f d\Omega \ ,\\
            \text{with} \quad f &= f(c, \nabla c, \nabla^2 c, \ldots)\, .
        \end{aligned}
    \end{equation}
    The local free energy density $f$ can be expanded in a Taylor series around $f_0$, the free energy of a uniform system or the bulk free energy density. Thus, the bulk free energy density $f_0$ represents the `\textit{interaction of different components in a homogeneous system}' \cite{Wu2022}. The thermodynamically relevant expression is logarithmic (Helmholtz) and is given by \cite{LeeHuhJeongShinYunKim2014},
    \begin{equation}
        f_0(c) = \frac{1}{N_a}\{\omega c (1-c) + RT \left[(1-c)\ln{(1-c)} + c \ln{c}\right]\}
        \label{eq:1-free-energy}\ ,
    \end{equation}
    with $N_a$ the Avogadro number, $\omega$ the regular solution parameter, $R$ the perfect gas constant, $T$ the temperature and $c$ the composition (more informations in \cite{LeeHuhJeongShinYunKim2014}).\\
    In practice, the thermodynamically relevant expression of the free energy density of an homogeneous system is replaced by a polynomial approximation of degree four,
   \begin{equation}
        f_0(c) = w g(c) = w (c-c_\alpha)^2 (c-c_\beta)^2\ ,
   \end{equation}
   where $g(c)$ is the polynomial function, $w$, $c_\alpha$ and $c_\beta$ are constants chosen to fit the positions of the minima and the curvature at the minima, as shown in \autoref{fig:1-free-energy}.
    \begin{figure}[H]
        \centering
        \includegraphics[width=0.4\textwidth]{chap1/1-free-energy.png}
        \caption{Bulk free energy density $f_0$ of a binary mixture and its quartic approximation. \cite{LeeHuhJeongShinYunKim2014}}
        \label{fig:1-free-energy}
    \end{figure}
    After some algebraic manipulations and assuming the system to be centrosymmetric, and thus isotropic, the free energy functional can be rewritten as,
    \begin{equation}
        \mathcal{F}(c) =\int_\Omega f_0(c) + \frac{\kappa}{2} |\nabla c|^2 d\Omega \ ,
    \end{equation}
    with $\kappa$ the gradient energy coefficient. A more thorough derivation of the total free energy functional of a nonuniform isotropic binary system is available in Moelans et al.~\cite{MoelansBlanpainWollants2008} or in Lee et al.~\cite{LeeHuhJeongShinYunKim2014}.\\
    This expression of the free energy functional is known as the Ginzburg-Landau \cite{LeeHuhJeongShinYunKim2014,Voorhees2018,MoelansBlanpainWollants2008} free energy functional. 
    Thermodynamics informs us that for a closed system, at fixed temperature and volume, to be in equilibrium, its free energy must be minimized. Thus, an equilibrium state is reached when the composition field $c$ is such that it extremizes the free energy functional, 
    \begin{equation}
        \delta \mathcal{F} = 0\ .
    \end{equation}
    After some variational calculus, this leads to the Euler-Lagrange equation \cite{Voorhees2018},
    \begin{equation}
        \frac{\partial f_0}{\partial c} - \kappa \nabla^2 c = 0\ .
        \label{eq:1-euler-lagrange}
    \end{equation}
    The solution $c$ to this equation extremizes the free energy functional. However, there are no constraint on the average value of $c$ in the system, i.e. the total mass is not conserved. To ensure mass conservation, a constraint is added through a Lagrange multiplier $\mu$ which leads to,
    \begin{equation}
        \mu = \frac{\delta \mathcal{F}}{\delta c} = \frac{\partial f_0}{\partial c} - \kappa \nabla^2 c
    \end{equation}
    the generalized chemical potential \cite{Voorhees2018}.\\
    Some authors skip the discussion on the constraint and directly define the chemical potential as the variational derivative of the free energy functional with respect to the order parameter $c$. 
    The order parameter $c$ is a conserved quantity and satisfies the continuity equation,
    \begin{equation}
        \frac{\partial c}{\partial t} + \nabla \cdot \mathbf{J} = 0\ ,
    \end{equation}
    where $\mathbf{J}$, the diffusion flux, is given by,
    \begin{equation}
        \mathbf{J} = -M \nabla \mu \ ,
        \label{eq:1-flux}
    \end{equation}
    where the the gradient of the chemical potential $\nabla \mu$ is the driving force of the diffusion, the thermodynamic force, and $M$ is the mobility function \cite{Voorhees2018,MoelansBlanpainWollants2008}. The mobility function $M$ can be a function of the order parameter $c$ as it stems from the combined effect of atomic mobilities of the system constituents \cite{Voorhees2018,MoelansBlanpainWollants2008}.
    This leads to the Cahn-Hilliard equation,
    \begin{equation}
        \frac{\partial c}{\partial t} = \nabla \cdot \left[ M \nabla \left( \frac{\partial f_0}{\partial c} - \kappa \nabla^2 c \right) \right]\ ,
    \end{equation}
    with $M$ the mobility parameter which is function of the atomic mobilities of the constituents of the system.
\subsection{Applications}
    Phase-field modeling has been used in a wide range of applications from material sciences to biology and biomedical sciences (tumor growth \cite{XiaoFengShi2023}).\\
    In materials science, different scales can be identified. At the mesoscale, fracture mechanics \cite{XueChengLeiWen2022}, fluid flows \cite{Kim2009}, at the microscale, solidification \cite{KimKim2005}, grain growth and spinodal decomposition \cite{CartaladeYounsiRégnierSchuller2014}, coarsening kinetics \cite{ZhuChenShenTikare1999,KönigRonsinHarting2021}, can be studied following the phase-field formalism. Indeed, provided a careful design of the free energy functional $\mathcal{F}$ of the studied system, different multiphysics problem can be studied. Following this downward trend in system scale, the phase-field formalism has been eventually nanoscale systems.\\
    Chockalingam et al.~studied the sintering mechanism of silver nanoparticles by exploring the phase field formalism. They were able to compare their numerical results with experimental observations which highlighted the strenghts of the phase field formalism at simulating physics at the nanoscale \cite{ChockalingamKouznetsovaSluisGeers2016}. However, the considered model of sintering was limited to the case of 2D circular particles. Roy et al.~\cite{RoyVarmaGururajan2021} extended the study of the sintering mechanism to the case of 3D nanowires. They were able to produce results which were in good agreement with experimental observations in their idealised case of a free-standing nanowire configuration. The goal of the present work is to extend the results of Roy et al.~to a more complex case of nanowire configurations, nanowires deposited on a substrate. This configuration would account for the interaction of the nanowires with the flat substrate they are deposited on, thus completing the model of sintering of nanowires.
        
