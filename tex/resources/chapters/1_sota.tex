% This first chapter introduces the required theoretical tools needed to study the morphological instability of metallic nanowire. The first section is dedicated to the introduction of the concept of transparent conducting materials/electrodes (TCM/Es) and more specially metallic nanowire-based TCEs. Metallic nanowire networks are then compared to transparent conducting oxides, the industry leading transparent conducting materials nowadays. Afterwards, a review of the different type of electrical activation methods are discussed with a focus on thermal annealing which shed light on the thermal instability of such metallic nanonetworks. The thermal instability leads to the morphological instability of the nanowires which is introduced in the second section. 
This first chapter introduces the theoretical tools needed to study the morphological instability of metallic nanowires numerically. To do so, three concepts are discussed.\\
First, the notion of metallic nanowires (MNWs) by introducing the concept of transparent conducting materials/electrodes, their field of applications and, in the case of MNWs, the different electrical activation methods used in practice, with a focus on thermal annealing. \\
In addition, the concept of morphological instability is discussed through the introduction of the thermal instability that metallic nanowires suffer, especially Ag-MNWs. This section covers the works of Plateau-Rayleigh in their first theoretical description of the phenomenon. \\
Finally, a detailed description of phase field modeling theory, the model utilised in this study, is performed.
\section{Metallic nanowires}
    %First, the notion of metallic nanowires (MNWs) within the broader context of transparent conducting materials and electrodes. This section begins by defining TCMs, discussing their various field of applications and introducing the relevant properties and a means to compare different types of TCMs through a Figure of Merit (FoM). The focus then shifts to metallic nanowires, emphasizing their advantages over the currently most used TCM, Indium-Tin Oxides (ITOs). Different electrical activation methods are then reviewed, with a particular focus on thermal annealing, which shed light on the thermal instability that takes place at the nanowire scale.
\subsection{Transparent conducting materials}
    Transparent conducting materials (TCMs) constitute a special class of materials which exhibits both high electrical conductivity and optical transparency \cite{Bardet2021}. In practice, materials which have high electrical conductivity, such as metals, are optically opaque whereas optically transparent materials such as glass, polymers or certain metal oxides are electrical insulators. This tradeoff between electrical conductivity and optical transparency can be understood through energy bandgap theory in solid state physics.\\
    Indeed, electrical insulators are often optically transparent because of their large bandgap 
    which allow photons from the visible spectrum to pass through without leading to electronic transitions and thus absorption. On the other hand, metals have a high density of states at the Fermi level which leads to high electrical conductivity but also to high absorption of photons in the visible spectrum.
    \begin{figure}[H]
        \centering
        \includegraphics[width=.8\linewidth]{chap1/1-energy-band}
        \caption{Energy band diagram of (a) electrical insulators (b) metals.}
        \label{fig:1-bandgap}
    \end{figure}
    Since the discovery of wide bandgap (above $3.1\,\text{eV}$) semiconductors in the 1950s, TCMs have played an important role in many industrial applications such as solar cells, transparent heaters, touch screens, OLED displays and many others. TCMs are typically used as electrodes (TCEs or TEs) in their various applications, where the dual optimization of the electrical conductivity and the optical transparency is crucial. For instance, in solar cell applications, the optical transparency is defined as the transmittance in the visible spectrum such as $550\,\text{nm}$.
    \begin{figure}[H]
        \centering
        \includegraphics[width=0.5\textwidth]{chap1/1-haacke-fom}
        \caption{Haacke Figure of Merit (FoM) for different families of TCMs.\cite{LagrangeLangleyGiustiJimenezBrechetBellet2015}}
        \label{fig:1-haacke-fom}
    \end{figure}
    In 1976, Haacke proposed the following Figure of Merit (FoM) to compare different families of TCMs. The FoM is defined as,
    \begin{equation*}
        \text{FoM} = \frac{T^{10}}{R_s}
    \end{equation*}
    The exponent is chosen such that the FoM is maximized for transmittance higher than $90\%$. This leads to the isolines in \autoref{fig:1-haacke-fom}. Thus, the region of interest for TCMs lies in the upper left corner, at high FoM values. One can see that two families of TCMs lie in this region of interest, transparent conducting oxides (TCOs), especially Indium-Tin oxides (ITOs), and metallic nanowire networks, especially Silver-based one (Ag-MNWs).\\
    Transparent conducting oxides (TCOs) and more particularly Indium-Tin Oxides (ITOs) are the industry leading choice for TCM application. They are heavily doped wide bandgap (above $3.1\,\text{eV}$) semiconductors. After more than 60 years of research, ITOs have now reached process stability and maturity. However, with the fast growing technological market and the consumers needs, new requirements for TCMs have surfaced. Indeed, the market now orients towards flexible electronics, therefore the TEs required for these new devices must be able to bend without losing their function. Unfortunately, one of the major drawbacks of ITOs is their brittleness. In addition to their mechanical limitations, with the fast growing demand of TEs, the high scarcity Indium and the high production cost in their synthesis lead to the search of alternatives which would solve theses issues.\\
    Thus the emerging TCMs must be low cost, use earth abundant materials, with competing conductivity and transparency to ITOs. This leads to the introduction of the center topic of this thesis, metallic nanowire network TCMs.

    Metallic nanowires (MNWs) are part of the emerging TCMs and are amongst the most promising especially polyol-grown Silver (Ag) nanowires. Indeed, due to their intrinsically high electrical conductivity, tunable optical transparency and their relatively simple synthesis and deposition method, MNWs have the potential to replace ITOs. TCMs based on MNWs can be achieved by depositing a percolating network of MNWs onto an optically transparent substrate. Percolation refers to the formation of a continuous path of conductive material from one side to the other. Thus the electrical conductivity is dependent on the density of the deposited MNWs. However, the denser the network, the less transparent it becomes. This tradeoff is one of the key elements in the optimization of MNW-base TCMS.
\subsection{Electrical activation methods of Metallic Nanostructures}
    After depositing the MNWs, the network must be activated to ensure percolation since the contact resistance are initially high between each nanowires. This can be achieved by different means,
    \begin{itemize}
        \item Mechanical pressing,
        \item Electrical annealing,
        \item Thermal annealing.
    \end{itemize}
    The most common activation method is the latter, thermal annealing, which consists in heating the network to reduce the contact resistance. Indeed, thermal annealing leads to the desorption of organic residues and to local sintering, formation of welds between nanowires.
    \begin{figure}[H]
        \centering
        \includegraphics[width=0.8\textwidth]{chap1/1-thermal-annealing}
        \caption{(a) Evolution of the electrical resistance with respect to the temperature during thermal annealing during a thermal ramp of $15^\circ\,\text{min}^{-1}$. Figure adapted from Langley et al.\ \cite{Langley2014}.}
        \label{fig:1-thermal-annealing}
    \end{figure}
    \autoref{fig:1-thermal-annealing} shows the evolution of the electrical resistance of a AgNW network with respect to the temperature during thermal annealing. Three steps can be identified from it.
    \begin{enumerate}
        \item Desorption of the organic residues.
        \item Local sintering of the nanowires.
        \item Spheroidization of the nanowires.
    \end{enumerate}
    As can be seen from the figure, when temperature crosses a certain threshold (around $300^\circ - 350^\circ$ for AgNW networks), the resistance increases rapidly making the network no longer conductive. SEM images shows that the nanowires transform into a series of chunks. This phenomenon is referred as the morphological instability of the nanowires.


\section{Morphological instability}
    In the literature, this phenomenon is often referred to as Plateau-Rayleigh instability, dewetting or spheroidization. This section focuses on the historic of the description of such instabilities and the discrepancies between the theoretical predictions and experimental observations in metallic nanowires.
\subsection{Theoretical description}
        In 1873, Joseph Plateau conducted a series of experiments which led to the observations of instability in liquid jets. He observed that a `vertically falling stream of water' breaks up into a series of droplets if the length of the stream exceeds a multiple of the initial diameter of the stream. Later, in 1878, Lord Rayleigh provided a theoretical model to explain the phenomenon observed by Plateau. His arguments were related on the minimization of surface energy. His model confirmed Plateau's observations and predicted that the jet would break up into droplets with a characteristic wavelength $\lambda = 9.016 R_0$. This phenomenon is now known as Plateau-Rayleigh instability.
        \begin{figure}[H]
                \centering
                \includegraphics[width=0.5\textwidth]{chap1/1-plateau-rayleigh.png}
                \caption{Plateau-Rayleigh instability in a liquid jet. Adapted from \cite{Rutland1970}}
                \label{fig:plateau-rayleigh}
        \end{figure}
        Following this description, Nichols and Mullins \cite{NicholsMullins1965} worked on the stability of solids of revolution. By considering surface-driven mass transport, they were able to predict spheroidization of cylindrical rods. McCallum et al. \cite{McCallum1996} extended this work for the case of cylindrical rods deposited on a substrate with varying contact angle. They found that the presence of the substrate provides a stabilizing effect to the morphological instability phenomenon.\\
        The wavelength of the instability predicted by McCallum is of the same order of magnitude as the one predicted by Plateau-Rayleigh for free-standing jets but for contact angle below $\theta=\pi$, which correspond to the free-standing case, the wavelength is slightly larger and the growth rate is smaller. Thus, the presence of the substrate is stabilizing.
\subsection{Application to metallic nanowires}
        Langley et al. \cite{Langley2014} found that the distance between the nanoparticles after spheroidization is of the same order of magnitude as the one predicted by Plateau-Rayleigh but is slightly larger. Stemming from this observation and the work of McCallum, Balty et al. \cite{BaltyBaretSilhanekNguyen2024} showed that the predictions provided by McCallum are in better agreement with the experimental observations.
        \begin{figure}[H]
                \centering
                \includegraphics[width=0.5\textwidth]{chap1/1-balty.png}
                \caption{Mean wavelength of the instability with respect to the initial radii of AgNWs. \cite{BaltyBaretSilhanekNguyen2024}}
                \label{fig:1-plateau-mccallum}
        \end{figure}
        Thus, McCallum model provides a good theoretical framework to understand the morphological instability of metallic nanowires which paves the way towards stabilization stategies and a better understanding of the mechanism at play at that scale. However, the model is limited to the description of infinitely long cylinders and cannot take into account the effect of a junction between two nanowires or the effect of their pentagonal cross-section.\\
        Numerical simulations are then required to deepen the understanding of the phenomenon and to provide a more accurate description of the morphological instability of metallic nanowires which is the focus of this study.

\section{Phase field modeling theory}
    \subsection{Theoretical context}
    In materials science, many important processes take place at the mesoscale. The mesoscale is defined as the in-between nanoscale and macroscale. Mesoscale processes can impact on the measured macro-properties of a system. Thus, accurate model which simulate the physics at that scale are required to understand it.\\
    Initially, the sharp-interface approach was used to study the physics at that scale. However, many processes at the mesoscale suffer from this mathematical definition and become almost intractable using this approach.\\
    Indeed, using a sharp-interface approach, consider the case of solidification.
    \begin{figure}[H]
        \centering
        \includegraphics[width=0.5\textwidth]{chap1/1-free-boundary.png}
        \caption{Sharp interface approach to solidification.}
        \label{fig:1-sharp-interface}
    \end{figure}
    To solve the problem, two diffusion equation must be solved, one inside the solid phase and the other inside the liquid phase. However, both equations depend on the position of the interface which depends on the composition in each phase. This type of problem is referred as a free boundary problem.
    \begin{equation}
        \begin{aligned}
            &\frac{\partial c^i}{\partial t} = D^i \nabla^2 c\quad,\quad i=\alpha, \beta\\
            &\frac{\partial R}{\partial t} = D \frac{\partial c}{\partial r}\Bigl|_{r=R(t)}\\
            &\mu^\alpha(c^\alpha_{int}) = \mu^\beta(c^\beta_{int})
        \end{aligned}
    \end{equation}
    The first equation is related to chemical diffusion and is referred to Fick's second law \cite{Gottstein2004}. The second equation tracks the motion of the interface. Finally, the third equation describes the thermodynamic constraint which states that both phases are in equilibrium at the interface. The problem with this approach is that the interface is not well defined and the equations are coupled.\\
    A solution to this problem was proposed by Langer \cite{Langer}. He proposed a description using a single equation which holds true in the entire domain. To do so, the sharp interface is approximated by a diffuse interface. This approach was first thought to be too complex to be useful. However, with the technological advances, diffuse-interface approach became the standard when studying microstructural evolution, in particular phase-field modeling.
    \begin{figure}[H]
        \centering
        \includegraphics[width=0.5\textwidth]{chap1/1-diffuse-interface.png}
        \caption{Diffuse interface approach.}
        \label{fig:1-diffuse-interface}
    \end{figure}
    In this context, the microstructure is described by a set of continuous fields which varies smoothly across the interface. Within each phase, the field has the same values and meaning as in the sharp interface approach. The position of the free interface can be retrieved through contours of constant values of the field variable. In addition, no constraint are required at the interface.
    \begin{figure}[H]
        \centering
        \includegraphics[width=0.3\textwidth]{chap1/1-phase-field.png}
        \caption{Example of a two phase microstructure. \cite{LeeHuhJeongShinYunKim2014}}
        \label{fig:1-phase-field}
    \end{figure}
\subsection{Fundamental principles and Formulation}
    The microstructure is described by a set of continuous fields \cite{}, i.e. the phase-field variables or order parameter. These variables can either be conserved or non-conserved depending on the phase-field model used. Conserved order parameter often refers to the local composition whereas non-conserved order parameter often refers to crystal structure or to the phase of a composition (e.g. solid-liquid).\\
    The driving force of microstructural dynamics is the minimization of the free energy of the system \cite{} which can be written as,
    \begin{equation}
        F = F_{bulk} + F_{int} + F_{source}
    \end{equation}
    with $F_{bulk}$ the free energy associated to the bulk of the system, $F_{int}$ the free energy associated to interfacial interactions and $F_{source}$ the free energy associated to additional sources of energy such as elastic strains, electromagnetic fields, and so on \cite{}.\\
    Classically, thermodynamic properties are assumed homogeneous throughout the system. However, in the case of phase-field modeling, the system is considered `\textit{nonuniform}', i.e. `\textit{a system having a spatial variation in one of its intensive scalar properties, such as composition or density}'\cite{CahnHilliard1958}. Its free energy is then given by a functional of the phase-field variables,
    \begin{equation}
        \begin{aligned}
            \mathcal{F}(c) &= \int_{\Omega} f d\Omega\\
            f &= f(c, \nabla c, \nabla^2 c, \ldots)
        \end{aligned}
    \end{equation}
    The local free energy density $f$ can be expanded in a Taylor series around $f_0$, the free energy of a uniform system. Thus, the bulk free energy density $f_0$ represents the `\textit{interaction of different components in a homogeneous system}' \cite{Wu2022}. The thermodynamically relevant expression is logarithmic (Helmholtz). However, in practice it is approximated using a quartic polynomials with minima at the equilibrium composition.
    \begin{figure}
        \centering
        \includegraphics[width=0.5\textwidth]{chap1/1-free-energy.png}
        \caption{Bulk free energy density $f_0$ of a binary mixture and its quartic approximation. \cite{LeeHuhJeongShinYunKim2014}}
        \label{fig:1-free-energy}
    \end{figure}
    After some algebraic manipulations and assuming the system to be centrosymmetric, the free energy functional can be written as,
    \begin{equation}
        \mathcal{F}(c) =\int_\Omega f_0(c) + \frac{\kappa}{2} |\nabla c|^2 d\Omega
    \end{equation}
    with $\kappa$ the gradient energy coefficient.\\
    This expression of the free energy functional is known as the Ginzburg-Landau\cite{} free energy functional. Thus, an equilibrium state is reached when the composition field $c$ is such that it extremizes the free energy functional. Variational calculus leads to Euler-Lagrange equation,
    \begin{equation}
        \frac{\partial f_0}{\partial c} - \kappa \nabla^2 c = 0
    \end{equation}
    The solution of this equation extremizes the free energy functional. However, there are no constraint on the average value of $c$ in the system, i.e. the total mass is not conserved. To ensure mass conservation, a constraint is added through a Lagrange multiplier $\mu$ which leads to,
    \begin{equation}
        \mu = \frac{\delta \mathcal{F}}{\delta c} = \frac{\partial f_0}{\partial c} - \kappa \nabla^2 c
    \end{equation}
    the generalized chemical potential.\\
    The order parameter $c$ is a conserved quantity and satisfies the continuity equation,
    \begin{equation}
        \frac{\partial c}{\partial t} + \nabla \cdot \mathbf{J} = 0
    \end{equation}
    where $\mathbf{J}$, the diffusion flux, is given by,
    \begin{equation}
        \mathbf{J} = -M \nabla \mu
    \end{equation}
    This leads to the Cahn-Hilliard equation,
    \begin{equation}
        \frac{\partial c}{\partial t} = \nabla \cdot \left[ M \nabla \left( \frac{\partial f_0}{\partial c} - \kappa \nabla^2 c \right) \right]
    \end{equation}
    with $M$ the mobility parameter which is function of the atomic mobilities of the constituents of the system.
\subsection{Applications}
    Phase-field modeling has been used in a wide range of applications from material sciences to biology and biomedical sciences (tumor growth).\\
    In material science, different scales can be identified. At the mesoscale, fracture mechanics \cite{}, fluid flows \cite{}, at the microscale, solidification \cite{}, grain growth \cite{}, spinodal decomposition \cite{}, coarsening kinetics \cite{}, and at the nanoscale, nanodots sintering \cite{sintering} and eventually nanowire breakups \cite{RoyVarmaGururajan2021} which inspired the present work.
        
