\section*{Context}
Transparent electrodes (TEs) are essential in modern optoelectronic applications such as touchscreens, OLED displays, solar cells, and wearable electronics. 
Recently, with the increasing demand for flexible electronics and for low-cost production, new class of transparent conducting materials (TCMs) are required to replace the industry standard, Indium-Tin Oxide (ITO), which suffers from their brittleness and scarcity \cite{Bellet2017,MauryaGalvanGautamXu2022, Nguyen2022}. 
In this context, metallic nanowires (NWs), especially polyol-grown Silver (Ag) nanowires, have emerged as a promising alternative thanks to their remarkable opto-electronic properties, low cost, and ease of manufacturing (e.g. the polyol process) allowing industrial scaling \cite{SohnParkOhKangKim2019,Sun2003,Bellet2017,JiuSuganuma2016}. 
However, metallic nanowires suffers from a limited thermal stability \cite{LagrangeLangleyGiustiJimenezBrechetBellet2015}. Above a certain temperature threshold (typically in the range of $300^\circ$-$350^\circ$ for AgNW), the nanowires undergo a transformation into a series of equally-spaced chunks which leads to the loss of conductivity of the network. 
This phenomenon is known as morphological instability and is a major limitation to technical applications where high-temperature annealing is required.
%\\This morphological instability can be found under different names in the literature, such as Plateau-Rayleigh instability, dewetting, spheroidization, ovulation or liquid-like instability \cite{Langley2014,NicholsMullins1965,Nichols1976,NaikDasPrewettRaychaudhuriChen2012, LianWangSunYuEwing2006,Plateau1873,Rayleigh1878}. Plateau-Rayleigh instability refers to the instability that occurs in liquid jet, i.e. vertically falling liquid columns. This phenomenon was first described by Joseph Plateau in 1873 \cite{Plateau1873} and later confirmed and theoretically described by Lord Rayleigh in 1878 \cite{Rayleigh1878}. 
%Langley et al.~\cite{Langley2014} have however shown descripancy between 
\section*{Objectives}
The main goal of this thesis is to investigate the morphological instability of metallic nanowires numerically using the phase-field formalism, specifically the Cahn-Hilliard equation with variable mobility \cite{CahnHilliard1958, Cahn1959,Cahn1961,LeeHuhJeongShinYunKim2014,ZhuChenShenTikare1999,Langer1975}. The goals are to further assess the relevance of the phase-field formalism by reproducing and extending on the work of Roy et al.~\cite{RoyVarmaGururajan2021}. A dimensional analysis of some key parameters of the morphological instability is performed. In addition, the impact of geometric features is studied to further approach the real structure of metallic nanowires. The phase-field formalism is then extended to consider the case of metallic nanowires deposited on a substrate. The Smoothed Boundary Method (SBM) \cite{YuChenThornton2012} is used to model the boundary conditions at the nanowire-substrate interface. A parallel between the numerical results and the theoretical predictions of Nichols and Mullins \cite{NicholsMullins1965,Nichols1976} and McCallum et al. \cite{McCallumVoorheesMiksisDavisWong1996} is then drawn out to further validate the approach.
\newpage
\section*{Outline}
This thesis is divided into five chapters.

The state of the art as well as the theoretical concept needed for the understanding of the work are presented in \autoref{chap:1-sota}. 

The methodology employed in the study is described in \autoref{chap:2-methodology} where the implementation of the phase field formalism is discussed. Numerical experiments are performed in order to validate the presented implementation and a further discussion on a first idealised description of a nanowire is performed. 

In \autoref{chap:3-results}, the implemented model is used to extend the study on free-standing nanowires by providing a thorough non-dimensional analysis of the key parameters of the morphological instability. A first consideration of the initial morphology of the nanowires is then performed in order to assess possible discrepancies between the idealised structure of a nanowire and a closer representation of the real structure.

Then, in \autoref{chap:4-substrate}, the model is further extended to consider the influence of a flat substrate on which the nanowires have been deposited. The theory of the Smoothed Boundary Method (SBM) is introduced and discussed in the case of the phase-field formalism, deriving a contact angle boundary condition and discussing its relevance with the literature. The results for the case of a single nanowire on a substrate are presented and compared to the theoretical predictions of Nichols and Mullins. 

Finally, \autoref{chap:concl} concludes the thesis by summarizing the findings and proposing directions for future work.