In the literature, this phenomenon is often referred to as Plateau-Rayleigh instability, dewetting or spheroidization. This section focuses on the historic of the description of such instabilities and the discrepancies between the theoretical predictions and experimental observations in metallic nanowires.
\subsection{Theoretical description}
        In 1873, Joseph Plateau conducted a series of experiments which led to the observations of instability in liquid jets. He observed that a `vertically falling stream of water' breaks up into a series of droplets if the length of the stream exceeds a multiple of the initial diameter of the stream. Later, in 1878, Lord Rayleigh provided a theoretical model to explain the phenomenon observed by Plateau. His arguments were related on the minimization of surface energy. His model confirmed Plateau's observations and predicted that the jet would break up into droplets with a characteristic wavelength $\lambda = 9.016 R_0$. This phenomenon is now known as Plateau-Rayleigh instability.
        \begin{figure}[H]
                \centering
                \includegraphics[width=0.5\textwidth]{chap1/1-plateau-rayleigh.png}
                \caption{Plateau-Rayleigh instability in a liquid jet. Adapted from \cite{Rutland1970}}
                \label{fig:plateau-rayleigh}
        \end{figure}
        Following this description, Nichols and Mullins \cite{NicholsMullins1965} worked on the stability of solids of revolution. By considering surface-driven mass transport, they were able to predict spheroidization of cylindrical rods. McCallum et al. \cite{McCallum1996} extended this work for the case of cylindrical rods deposited on a substrate with varying contact angle. They found that the presence of the substrate provides a stabilizing effect to the morphological instability phenomenon.\\
        The wavelength of the instability predicted by McCallum is of the same order of magnitude as the one predicted by Plateau-Rayleigh for free-standing jets but for contact angle below $\theta=\pi$, which correspond to the free-standing case, the wavelength is slightly larger and the growth rate is smaller. Thus, the presence of the substrate is stabilizing.
\subsection{Application to metallic nanowires}
        Langley et al. \cite{Langley2014} found that the distance between the nanoparticles after spheroidization is of the same order of magnitude as the one predicted by Plateau-Rayleigh but is slightly larger. Stemming from this observation and the work of McCallum, Balty et al. \cite{BaltyBaretSilhanekNguyen2024} showed that the predictions provided by McCallum are in better agreement with the experimental observations.
        \begin{figure}[H]
                \centering
                \includegraphics[width=0.5\textwidth]{chap1/1-balty.png}
                \caption{Mean wavelength of the instability with respect to the initial radii of AgNWs. \cite{BaltyBaretSilhanekNguyen2024}}
                \label{fig:1-plateau-mccallum}
        \end{figure}
        Thus, McCallum model provides a good theoretical framework to understand the morphological instability of metallic nanowires which paves the way towards stabilization stategies and a better understanding of the mechanism at play at that scale. However, the model is limited to the description of infinitely long cylinders and cannot take into account the effect of a junction between two nanowires or the effect of their pentagonal cross-section.\\
        Numerical simulations are then required to deepen the understanding of the phenomenon and to provide a more accurate description of the morphological instability of metallic nanowires which is the focus of this study.