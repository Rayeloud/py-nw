In the literature, this phenomenon is often referred to as Plateau-Rayleigh instability, dewetting or spheroidization. This section focuses on the historical description of such instabilities and the discrepancies between the theoretical predictions and the experimental observations in metallic nanowires.
\subsection{Theoretical description}
        In 1873, Joseph Plateau conducted a series of experiments which led to the observations of instability in liquid jets \cite{Plateau1873}. He observed that a \textit{`vertically falling stream of water'} breaks up into a series of droplets if the length of the stream exceeds a multiple of the initial diameter of the stream. Later, in 1878, Lord Rayleigh provided a theoretical model to explain the phenomenon observed by Plateau\ \cite{Rayleigh1878}. His arguments were related on the minimization of surface energy. His model confirmed Plateau's observations and predicted that the jet would break up into droplets with a characteristic wavelength of $\lambda = 9.016\,R_0$, with $R_0$ the initial radius of the jet. This phenomenon is now known as Plateau-Rayleigh instability.
        \begin{figure}[H]
                \centering
                \includegraphics[width=0.7\textwidth]{chap1/1-plateau-rayleigh.png}
                \caption{Plateau-Rayleigh instability in a liquid jet. Adapted from\ \cite{RutlandJameson1971}}
                \label{fig:plateau-rayleigh}
        \end{figure}
        Following this description, Nichols and Mullins\ \cite{NicholsMullins1965} worked on the stability of solids of revolution. By considering surface-driven mass transport as the primary driving force, they developed a numerical method to compute \textit{`the kinetics of the shape changes of any solid of revolution'}. They found that the solution to a cylindrical rod is a series of equally-spaced spheres along the direction of the rod, the spheroidization of the rod.\\%their model was able to predict the spheroidization of cylindrical rods.
        McCallum et al.\ \cite{McCallumVoorheesMiksisDavisWong1996} extended this work for the case of cylindrical rods deposited on a substrate with varying contact angle. Their work showed that the presence of the substrate provides a stabilizing effect to the morphological instability phenomenon. Indeed, their work showed a decreasing trend of the non-dimensional growth rate $\sigma_m$ with respect to the contact angle $\alpha$ modeling the presence of a substrate. The function relating $\sigma_m$ to the contact angle $\alpha$ is represented in \autoref{fig:sigma_m_mccallum}.
        \begin{figure}[H]
                \centering
                \includegraphics[width=0.8\textwidth]{chap1/1-mccallum-sigma.png}
                \caption{Plot of (a) the maximum nondimensional growth rate $\sigma_m$ with respect to the contact angle $\alpha$ (b) the stability regions with respect to the contact angle $\alpha$. The solid line represents the value of $k_c^2$, the critical nondimensional wavenumber associated to a nondimensional growth rate $\sigma=0$. The dashed line represents the value of $k_m^2$, the nondimensional wavenumber associated to the maximum nondimensional growth rate $\sigma = \sigma_m$ \cite{McCallumVoorheesMiksisDavisWong1996}.}
                \label{fig:sigma_m_mccallum}
        \end{figure}
        The wavelength of the instability predicted by McCallum is of the same order of magnitude as the one predicted by Plateau-Rayleigh for free-standing jets, however for contact angle below $\theta=\pi$ corresponding to the free-standing case, the wavelength is slightly larger. Since it is inversely proportional to the wavenumber $k$, and the growth rate is smaller. Thus, the presence of the substrate is stabilizing.
\subsection{Application to metallic nanowires}
        In their experimental work, Langley et al.\ \cite{Langley2014} found that the distance between the nanoparticles after spheroidization is of the same order of magnitude as the one predicted by Plateau-Rayleigh but is slightly larger. Inspired by observations and by the work of McCallum, Balty et al.\ \cite{BaltyBaretSilhanekNguyen2024} showed that the predictions provided by McCallum are in better agreement with the measurements in an extended series of experiments summarized in \autoref{fig:1-plateau-mccallum}.
        \begin{figure}
                \centering
                \includegraphics[width=0.48\textwidth]{chap1/1-balty.png}
                \caption{Fitted mean instability wavelength with respect to the initial mean radii of AgNWs samples. The grey area represents the confidence interval of the fitted curve. McCallum theory provides a prediction closer to the experimental results than Plateau-Rayleigh's\ \cite{BaltyBaretSilhanekNguyen2024}.}
                \label{fig:1-plateau-mccallum}
        \end{figure}
        Thus, McCallum model provides a good theoretical framework to understand the morphological instability of metallic nanowires which paves the way towards stabilization strategies and a better understanding of the mechanism at play at that scale.\\
        However, the model is limited to the description of idealized infinitely long truncated cylinders, and cannot take into account the effect of a junction between two nanowires, the effect of the length/radius ratio, the effect of the crystalline nature of the nanowire which yields a pentagonal cross-section, and more.\\\\
        Numerical simulations are thus required to deepen the understanding of the phenomenon and to provide a more accurate and complete description of the morphological instability of metallic nanowires, which is the focus of this study. There exists several modeling theory one can utilize to model physics at the mesoscale, the scale encompassing the nanowire size, the scale in-between atomistic and macroscopic scales.\\
        Molecular Dynamics (MD) is a powerful numerical tool for simulating the motion of atoms in a molecular system. However, MD is computationally expensive and is limited to the study of small systems over small time scales with time steps on the order of the femtoseconds \cite{Tuckerman2000}. It is thus ill-suited for the study of the morphological instability and the resulting necking of nanowires.
        One of the most promising theory for modeling physical processes at the mesoscale, i.e. the scale in-between atomistic and macroscopic scales, is the Phase-field theory which gained popularity in the last decades for its versatility and ability to model complex geometries \cite{BartelsMosler2015}.